\documentclass[12pt, norsk, a4paper]{exam}
\usepackage[utf8]{inputenc}
\usepackage[pdftex]{color, graphicx}
\usepackage{graphicx, babel}
\usepackage{url}

\renewcommand{\labelenumi}{(\alph{enumi})}

% LaTeX er normalt streng når det gjelder linjebrytingen. Vi vil være litt
% mildere, særlig fordi norsk har så mange lange, sammensatte ord
\tolerance = 5000
\hbadness = \tolerance
\pretolerance = 2000

% For å få avsnitt uten innrykk:
\setlength{\parindent}{0pt}
\setlength{\parskip}{2ex plus 0.2ex minus 0.2ex}
\setlength\fboxsep{10pt}
\setlength\fboxrule{0.5pt}



\title{INF-1400 Introduksjon til Python}
% \author{Øyvind Holmstad, Martin Ernstsen, Erik Bræck Leer}
\date{16. januar, 2013}

\begin{document}
\maketitle

\section*{Informasjon}
Det er lurt å undersøke om du bruker riktig versjon av Python før du begynner,
boka bruker Python 3, så det kan bli lettere hvis du også bruker det (selv om
det er små forskjeller). LG-laben har Python 3 installert på Linux, men ikke
på Windows.

Hvor finner man informasjon om python? Et Google-søk på \emph{python} og navnet
på funksjonen eller det man ønsker å gjøre er ofte det enkleste. De første
treffene er som oftest til den offisielle dokumentasjonen på
\url{www.python.org} (som er veldig bra). En kort tutorial til python finnes
her: \url{http://hetland.org/writing/instant-python.html}.

Oppgavene under er ment å få deg i gang med python-programmering. Gjør så mange
som du rekker.

\section{Lister og løkker}
Oppvarming! Først, opprett ei tom fil med navnet \texttt{oppvarming.py}. Kjør
scriptet med å skrive \texttt{python oppvarming.py} i terminalen. I fila skal du
nå:

\begin{enumerate}

\item Lage ei for-løkke som går 100 ganger fra 0 til 99 og printer alle tallene
(ved bruk av Range).
\item Modisfisere for-løkken til å kun printe partall. 
\item Lage ei liste med stringene “panther”, “tiger”, “leopard”, “snow leopard”,
“windows vista”. Iterere over listen og print alle liste-elementene. 
\item Iterere over listen og print index fulgt av liste-element (hint:
enumerate). 
\item Fjerne siste element i lista
\item
Sortere lista alfabetisk
\end{enumerate}

\section{Moduler}
Bruk \texttt{oppvarming.py} eller lag ei ny fil om du
vil.
\begin{enumerate}  
\item Importer matematikkbiblioteket (math)
\item Bruk sinus, cosinus og/eller tangensmetoden(e).
\item Skriv en funksjon som bruker matematikkbiblioteket til å regne ut
kvadratroten av alle tallene fra 20 til 40, og skriver disse ut i terminalen. 
\end{enumerate}

\section{Dictionary og pass by value/reference}
\begin{enumerate} 

\item
Definer en funksjon \texttt{make\_phonebook()} som oppretter en dictionary og
fyller den mdn noen navn(key) og tilhørende telefonnummer(value). Funksjonen
skal returnere dictionaryen. 
\item Opprett en phonebook ved å kalle \texttt{make\_phonebook()}.  Iterer over
navnene(keys) og skriv ut. 
\item Iterer over numrene(values) og skriv ut. 
\item Definer en funksjon \texttt{print\_phonebook(phonebook)} som tar en
dictionary som argument, og skriver denne ut på formen ``\emph{navn} har
telefonnummer: \emph{12345678}''
\item Definer en funksjon \texttt{add\_ccodes(phonebook)} som tar en dictionary
som argument og legger til 47 foran alle telefonnumrene. Du kan anta at alle telefonnumre har 8 siffer. 
\item Kjør følgende kode. Hva skjer?  Hvorfor? 
\begin{verbatim}
    name_numbers = make_phonebook()
    print_phonebook(name_numbers)
    add_ccodes(name_numbers)
    print_phonebook(name_numbers)
\end{verbatim}
\end{enumerate}

\section{Klasser og metoder}
 I denne oppgaven skal vi lage en enkel
klasse som veldig enkelt simulerer en bankkonto. Opprett ei ny fil med navnet
\texttt{account.py}

\begin{enumerate}
\item Lag en klasse \texttt{Account} med metodene \texttt{deposit},
\texttt{withdrawal} og \texttt{status}. Accountklassen skal inneholde én variabel med navnet
\texttt{value}. Ved å bruke metodene \texttt{deposit} og \texttt{withdrawal} skal du henholdsvis
legge til og trekke fra en verdi fra \texttt{value}. \texttt{status}-metoden skal printe
nåværende verdi av value. 
\item Opprett ei ny fil hvor du importerer klassen du
nettopp har laget. Opprett et Account-objekt og bruk metodene til å legge til og
trekke fra kontoen. 
\end{enumerate}

\section{Guess a number!}
I denne oppgaven skal vi lage et spill hvor
datamaskinen plukker ut et tilfeldig tall mellom 0 og 100 og spilleren skal
gjette tallet. Gjetter spilleren for høyt får han tilbakemelding om dette. På
samme måte får spilleren vite om han har gjettet for lavt. Spillet avsluttes når
spilleren har gjettet riktig tall.

For å løse oppgaven skal du lage en klasse Guess som ved initialisering velger
et tilfeldig tall mellom 0 og 100. Klassen skal inneholde en metode
\texttt{guess()} som tar inn tallet spilleren gjetter på. \texttt{guess()} returner
-1, 0, eller 1, der -1 betyr at spilleren har gjettet for høyt, 1 betyr at
spilleren har gjettet for lavt, og 0 betyr at spilleren har gjettet riktig. Du
kan bruke \texttt{input("...")} for å lese inn tall fra terminalen.

\end{document}
